

\section{Virtual Worlds Over the Internet}
\label{sec:eval}

% 10 different camera locations, rotational view
% Village scene as well as island.
% 


%\begin{figure*}
%\subfigure[Volume-based aggregation ($\mu=0$)]{\includegraphics[width=3in]{fig/orchard-volume.png}}
%\subfigure[Instance-aware aggregation ($\mu=0.8$)]{\includegraphics[width=3in]{fig/orchard-iaa.png}}
%\caption{Visual improvement IAA provides on the orchard scene. Both scenes were
%simplified to XX MB with a branching factor of 10.}
%\label{fig:orchard-iaa}
%\end{figure*}


We demonstrate the benefits of our algorithms on three user-generated
worlds, each of which possesses
a significant degree of content coherence.  
\emph{Island} and \emph{Suburb} are user-generated scenes
described earlier in Section \ref{sec:bvh_incremental}. \emph{City}
was generated using the CityEngine tool. Table \ref{tab:optimization_benefits} summarizes the results.
For each scene, the table shows network transfer cost and rendering
cost using our algorithmic optimizations compared to an approach based on
traditional algorithms. Rendering cost is measured as the number of
draw calls needed to render all objects returned by the server. This
cost discounts atlasing, since atlasing reduces the draw calls for even a very
complex aggregate to just one.
The first four scenes have a high degree of explicit
instancing, \ie they have many identical objects. These scenes see a 36-80\%
reduction in download size using our algorithms, with negligible difference in visual quality.
The CityEngine scene does not
have any explicit instancing. Objects in the scene are also very simple and not
instanced. In this scene, our approach is able to achieve a 21\% reduction in
transfer cost, mostly by aggressive deduplication of similar meshes.
While this results in loss of small visual details, there is a substantial
reduction in both rendering and download costs.

{\setlength{\tabcolsep}{.1667em}
\begin{table*}
\centering
{\small
\begin{tabular}{lccccccc|ccc}
\toprule[2pt]
 & & & & \multicolumn{4}{c}{Download Size (MB)} &  \multicolumn{3}{c}{Draw Calls} \\
\cmidrule{5-8} \cmidrule{9-11}
 Scene         & Screenshot & Objects & \specialcell{Unique\\Objects} & Triangles  & \specialcell{No\\Optimizations} & \specialcell{With\\Optimizations} & Reduction & \specialcell{No\\Optimizations} & \specialcell{With\\Optimizations} & Reduction   \\ \hline
\addlinespace[0.1em]
% Tiles & \includegraphics[width=0.5in]{fig/grid-screenshot-client.png}  & 1600 & 100  & 8 million &  74   &  19 & 74\% & 347 & 309 & 11\%  \\
%\addlinespace[1em]
 Island & \includegraphics[width=0.5in]{fig/island-screenshot-client.png}  & 2362 & 236  & 13 million  &  349   &  224 & 36\%  & 1825 & 1529 & 16\% \\
\addlinespace[0.1em]
 Suburb & \includegraphics[width=0.5in]{fig/suburb_screenshot_client.png}  & 10,000 & 80  & 9 million  &  750   &  150 & 80\%  & 3470 & 2682 & 23\% \\
\addlinespace[0.1em]
 City  &  \includegraphics[width=0.5in]{fig/Point09-sa-dedup-screenshot-cityengine.png}  &60,000 & 60,000  & 16 million   &  712   &  563 & 21\%  & 34872 & 30211 & 13\% \\
\bottomrule[2pt]
\end{tabular}
}
\caption{Reduction in mesh download size and draw calls using
the optimizations in this paper. Draw calls are counted without
accounting for texture atlasing. The screenshot shows the scene
rendered with all optimizations enabled.}

\label{tab:optimization_benefits}
\vspace{-8pt}
\end{table*}
}

Next, we validate our techniques on a large, single-server world. 
The client and server are geographically separated
and connect over the Internet. 
The world consists of terrain and
10,000 unique textured tree models that are slight variations of three
user-generated base tree models. The trees all satisfy the instance scaling property
from Section \ref{sec:simplification_discussion}. Moreover, the trees have highly
instanced but very simple leaves. Quadric simplification is unsuitable for such models, while
IAS, especially its stochastic step, is much more effective. 
In total, the scene contains over 16 million
instances, 130 million triangles and occupies 3.6 GB on disk.
The system builds the BVH, deduplicates highly similar trees within aggregates, 
and generates their simplified models with no artist involvement.
As the client moves, the servers
dynamically update its LODs and tell it to download IAS
simplified models, some of which are IAA generated aggregates, from URLs in
the CDN. Figure \ref{fig:screenshot-teaser-fall} shows this world
rendered by our client at 1024x768 on a lower-end Intel
quad-core 3.1Ghz machine with an NVIDIA GeForce GT630 card. Without IAA and IAS, the
scene barely renders at 1-2 FPS, due to the large number and sizes of
models to be displayed. Using IAS and IAA, the client is able to render the scene
at 10-12 FPS, after downloading about 1.7 GB of data.

\begin{figure}
\centering
\includegraphics[width=0.4\textwidth]{fig/screenshots/screenshot-61.png}
\caption{A green alien moves through the city and sets its sights
on the next city at the horizon. All of the models and textures for this
scene are delivered from Amazon EC2 servers to a client on the opposite
coast of the US.}
\label{fig:screenshot-alt}
\vspace{-12pt}
\end{figure}


Finally, we demonstrate our algorithms on a large, multi-server world 
running on nine Amazon EC2 c1.xlarge servers. As before, 
the client connects to the world over the Internet.
Each instance runs a tiled copy of a 60,000 object urban scene.  
These objects were
generated with CityEngine and uploaded to the CDN as COLLADA models.
To diversify the world with user-generated content, we
replace 3\% of the models in the world with randomly chosen models
from the CDN, including space aliens, ships, and others. 
These 540,000 objects have over 144 million triangles
and over 6000 distinct textures. This content takes up 24 GB on
disk. On first entering the world, the client downloads 1.2 GB of data,
a 95\% reduction. 
Figure~\ref{fig:screenshot-teaser-space} shows this world rendered by our
client at 1920x1080. Figure~\ref{fig:screenshot-alt} shows another
view in the world. The client runs on a machine with an AMD FX-8150
eight-core CPU, 16 GB of RAM and an NVIDIA GeForce GTX 650Ti graphics
card.  IAS and IAA allow this complex scene to be rendered at frame
rates between 20 and 35 FPS.  Without IAS and IAA, our client is
overwhelmed by the size and number of models and textures to be
downloaded and displayed, rendering the scene at less than 1 FPS.
